\chapter{Aspectos metodológicos}  
%\addcontentsline{toc}{chapter}{\numberline{}Aspectos metodológicos}

\section{Metodología}

Historias de usuario, épicas, análisis de necesidades. Hablar sobre Agile, SCRUM, Kanban y cosas de estas.
Hacer anexo si son muchas.

\section{Tecnologías empleadas}

Electron, React, SQLite, GitHub, Webpack, MeCab.

\subsection{Licencia}

Al distribuir \textit{software}, es importante tener en cuenta la licencia que se utiliza al distribuirlo, ya sea en su forma binaria o compilada. Las licencias son textos que describen las formas en que los usuarios tienen permitido utilizar el \textit{software}. Estas licencias funcionan gracias a las leyes de \textit{copyright}.

Existen numerosos tipos de licencia, entre las cuales son especialmente relevantes en este trabajo las licencias de código abierto. Estas son licencias que permiten la libre distribución, uso o modificación del \textit{software}, así como de sus obras derivadas. Está claro que, para que un programa informático se pueda considerar de código abierto, es necesario dar acceso al código fuente al usuario, pues de lo contrario no podría modificarlo libremente como requiere la definición de \textit{código abierto}.

Que un programa se distribuya con una licencia de código abierto puede suponer grandes ventajas que afectan de manera positiva tanto al usuario como al equipo de desarrollo:

\begin{enumerate}
	\item 
\end{enumerate}

A grades rasgos, las licencias de código abierto se suelen dividir en dos categorías:

\begin{enumerate}
	\item \textbf{Licencias permisivas}: Son licencias que permiten la distribución de las obras derivadas bajo cualquier licencia que decida aplicar el autor de la obra derivada, siempre que se cumpla una serie de condiciones, como mantener la atribución de los autores originales. En esta categoría entran licencias como la MIT o Apache 2.0.
\end{enumerate}

\begin{enumerate}
	\item \textbf{Licencias restrictivas}: También conocidas como licencias \textit{copyleft}, son licencias que permiten la distribución de las obras derivadas únicamente bajo la misma licencia de origen o por otras licencias explícitamente permitidas por el autor original. En esta categoría entran licencias como las de GNU.
\end{enumerate}

Esta aplicación se ha decidido distribuir bajo la licencia GNU Affero General Public License 3.0. Es una licenica restrictiva con cláusulas específicamente diseñadas para asegurar que las obras derivadas distribuidas como sitios \textit{web} mantengan la licencia de origen.
