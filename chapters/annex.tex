\newpage{\pagestyle{empty}}
\markboth{Anexo}{Anexo}
\chapter*{Anexo}  
\addcontentsline{toc}{chapter}{\numberline{}Anexo}

\section{Historias de usuario}

\subsection{Épica 1}

Involucra la página principal para la lectura de textos.

\subsubsection{Historia de usuario 1.1}

Como estudiante de idiomas, quiero tener un diccionario incrustado en la aplicación para poder ver la definición de palabras que no conozco de forma cómoda.

\subsubsection{Historia de usuario 1.2}

Como estudiante de idiomas, quiero reconocer intuitivamente las palabras que todavía no he aprendido para prestarles especial atención y cuidado a la hora de leer.

\subsubsection{Historia de usuario 1.3}

Como estudiante de idiomas con nivel avanzado, quiero no tener que marcar como aprendida cada palabra que lea para que esto no sea un estorbo a la hora de leer un texto con muchas palabras que, aunque conozca, aún no han sido marcadas como tal en la aplicación.

\subsection{Épica 2}

Involucra el sistema de \textit{plugins}.

\subsubsection{Historia de usuario 2.1}

Como estudiante de un idioma con reglas ortográficas peculiares, quiero ser capaz de adaptar el funcionamiento de la aplicación a las necesidades específicas de esta lengua sin necesidad de modificar el código fuente. \todo ¿Para?

\subsubsection{Historia de usuario 2.2}

Como creador de un \textit{plugin}, quiero ser capaz de compartirlo con otros usuarios de una manera cómoda. \todo Revisar.

\subsection{Épica 3}

Involucra la importación de textos.

\subsubsection{Historia de usuario 3.1}

Como lector habitual de un sitio web en la lengua que estudio, quiero poder importar directamente textos de páginas web para no gastar el tiempo en copiarlo yo a mano y poder estudiar con contenidos que me resulten intrigantes.

\subsubsection{Historia de usuario 3.2}

Como lector de libros, quiero importar textos desde formatos comunes como EPUB o PDF para aumentar mi repertorio de lectura potencial y poder estudiar con contenidos que me resulten intrigantes.

\subsubsection{Historia de usuario 3.3}

Como lector habituado al formato de un libro, quiero que los textos se separen en páginas para tener una idea intuitiva de cuánto he leído y cuánto me queda por leer.

\subsection{Épica 4}

Involucra la configuración por defecto de idiomas populares.

\subsubsection{Historia de usuario 4.1}

Como estudiante de idiomas con limitados conocimientos de informática, quiero empezar a estudiar un idioma sin tener que pensar en su configuración para no equivocarme en el proceso ni agobiarme por no saber utilizar la aplicación.
