\chapter{Estado de la cuestión}  
\addcontentsline{toc}{chapter}{\numberline{}Estado de la cuestión}

\section{Marco teórico del trabajo}

En el siglo XX varios lingüistas se interesaron por la pregunta sobre la forma en que los adultos adquieren una segunda lengua \autocite{Ellis_2021}. 

Una de las aportaciones más significantes en la historia del estudio sobre la adquisición de la segunda lengua es el modelo monitor o la hipótesis de entrada\footnote{El Dr. Krashen originalmente bautizó como \textit{hipótesis de entrada} a una sola de las cinco hipótesis. Aunque el uso popular favorezca el uso del término \textit{hipótesis de entrada} para todo el conjunto, en el presente documento se lo denominará \textit{modelo monitor} para evitar confusiones.} propuesta por el lingüista Stephen Krashen \autocite{Krashen2003}. El modelo monitor es en realidad un conjunto de cinco hipótesis que ponen especial énfasis en la importancia de absorber contenido en la lengua objetivo, a la vez que consideran que la producción de contenido solo produce beneficios marginales \autocite{Krashen2003}. Dichas hipótesis son las siguientes:
\begin{itemize}
	\item \textbf{La hipótesis de adquisición-aprendizaje}: Dice que existen dos formas independientes de desarrollar las habilidades en un idioma. Por una parte, está el \textit{aprendizaje}, un proceso consciente: ocurre prestando atención al idioma, \textit{estudiando}, por ejemplo, sus reglas gramaticales Por otra parte, está la \textit{adquisición}, un proceso subconsciente, lo cual significa que la persona no se da cuenta de que aumentan sus habilidades.
	\item \textbf{La hipótesis del orden natural}: Dice que las distintas partes de un idioma se adquieren en un orden predecible. Krashen también describe tres propiedades de esta previsibilidad:
	\begin{enumerate}
		\item El orden no está dado por la complejidad de los conceptos a adquirir; puede ser que un ítem gramatical complejo suela adquirirse antes que otro simple. Esto dificulta el diseño de los libros de texto que no respetan el orden natural.
		\item El orden natural es inmutable, lo que significa que hacer más ejercicios o leer explicaciones más claras sobre un concepto gramatical no supone que vaya a adquirirse más rápido.
		\item El orden natural no es el que debería respetarse a la hora de enseñar un idioma, porque... \todo
	\end{enumerate}
	\item \textbf{La hipótesis del monitor}: Krashen se refiere como \textit{monitor} a la conciencia del hablante. La idea es la siguiente: antes de producir lenguaje, el cerebro filtra, de manera consciente, la frase, buscando errores gramaticales que pueden ser corregidos antes o después de pronunciar las palabras. Según esta hipótesis, esta \textit{monitorización} es la única función que tiene el pensamiento conciente, y en el largo plazo los beneficios que trae a la fluidez son marginales, resultando realmente útil tan solo en situaciones en que la mera adquisición de la lengua no sea suficiente para que el usuario interiorice por completo algunas propiedades del idioma.
	
	Krashen además explica que se deben cumplir tres condiciones para que pueda utilizarse el monitor: el hablante debe conocer la herramienta, conocer las reglas gramaticales del idioma, y contar con tiempo de aplicarla. Es poco probable que estas tres condiciones se den: sobre todo la tercera, pues la demora que conlleva aplicar este método ralentiza a niveles indeseados la conversación. Este es el motivo por el que Krashen recomienda utilizarlo en conversaciones de por sí lentas, o en la escritura.
	
	\item \textbf{La hipótesis de la entrada}: Esta hipótesis trata de responder a la pregunta de cómo la adquisición de un idioma ocurre realmente. La respuesta que propone es que es la comprensión de un texto la que produce la adquisición.
	
	Esta hipótesis depende en cierta medida de la hipótesis del orden natural, la cual, recuérdese, indica que existen elementos de un idioma que los estudiantes adquieren antes que otros. Se puede formular, entonces, que según los elementos del idioma que el estudiante ha adquirido hasta el momento, se le puede asignar un nivel numérico, como $i$. El valor de $i$ no es importante por sí mismo; tan solo es una nomenclatura para indicar que este nivel es más avanzado que otro hipotético nivel $i - 1$, y que también existe un nivel más avanzado como $i + 1$.
	
	Con esta reformulación, la hipótesis de entrada explica que la forma en que el estudiante pasa de tener un nivel $i$ a tener un nivel $i + 1$ es comprendiendo contenidos con elementos del nivel $i + 1$.
	
	Esto es posible de diversas maneras, como un estudio activo y consciente de dichos elementos, o presentar el idioma junto con contenidos extralingüísticos, como imágenes.
	
	Krashen, además, asegura que la hipótesis de la entrada funciona por causa-efecto. Es decir: si una persona absorbe y comprende contenido de nivel $i + 1$, la misma inevitablemente acabará adquiriendo el nivel $i + 1$.
	
	También predica que la mera producción de la lengua (es decir, hablar o escribir en el otro idioma) no afecta directamente a la adquisición de la lengua. Aun así, indirectamente sí puede ser de ayuda, ya que los efectos psicológicos y sociales de ser un hablante de la lengua facilitan tomar provecho de oportunidades para absorberla.
	
	\item \textbf{La hipótesis del filtro afectivo}: Esta hipótesis es una excepción al primer colorario de la hipótesis de la entrada. Según \todo, el cerebro contiene una parte denominada \textit{dispositivo de la adquisición de idiomas} \todo[Recalcar que este concepto de la gramática generativa de Noam Chomsky ha sido refutado por la ciencia europea, aunque en EE. UU. aún se trabaja]. Este dispositivo, que se encarga de la adquisición, puede ser bloqueado por el \textit{filtro afectivo}, causado por una situación psicológica desfavorable, como baja autoestima o ansiedad. \todo[¿Emociones fuertes no deberían ayudar?]
\end{itemize}

El modelo monitor fue propuesto hace al rededor de cinco décadas. Por supuesto, a lo largo de estos últimos años, se han realizado innumerables aportes al estudio sobre la adquisición de la segunda lengua que, si bien no contradicen del todo las ideas arriba mencionadas, sí expanden sobre ellas y aclaran matices importantes \autocite{https://doi.org/10.1111/flan.12552}. \todo[¿Cuáles?]

\todo[Crítica a Krashen: niveles conversacionales demasiado bajos]

\todo[Lingüística moderna: patrones del lengüaje y su reproducción. Interlengua]

En definitiva, aunque se han realizado modificaciones a la propuesta original de Krashen, los pilares de su teoría no han sido refutados de momento y, al contrario, han sido respaldados con pruebas empíricas.

\todo[Uso de los dispositivos electrónicos para el estudio de idiomas: estadísticas de Duolingo y similares]

\todo[Miles de caracteres en japonés]

\section{Trabajos relacionados}

La idea de crear una aplicación que haga del modelo monitor un método de estudio accesible no es del todo novedosa, ni tampoco reciente. A continuación se presentan una serie de trabajos que tienen unos objetivos similares a los que se tratan en el presente proyecto.

\subsection{LingQ}

LingQ es una plataforma web para la lectura extensiva, creada por Steve Kaufmann y Mark Kaufmann en 2007, con las ideas del lingüista Krashen en mente \autocite{LingQ}. La plataforma contiene una gran variedad de recursos para utilizar como entrada por aquellos estudiantes de las lenguas más populares \autocite{clara2025}, en forma de cuentos, cursos, libros, programas de radio, y otros.

\todo[Breve explicación de su uso]

Una característica de esta plataforma, que la distingue de otras más populares como Duolingo es su falta de estructura \autocite{Karasimos}; el usuario decide en todo momento qué contenidos leer, buscando una lectura que les interese, y permitiendo que los elementos lingüísticos a aprender se adquieran automáticamente mediante la lectura, como recomienda el modelo monitor.

LingQ es un servicio de pago; los usuarios deben pagar una cuota de mensual a bianual para tener acceso ilimitado a las funcionalidades \autocite{kabbasovna}.

El más extenso análisis sobre LingQ es el compartido por Cristina Rodríguez Pastor \autocite{Pastor_2022}. En él, la autora explica cómo la metodología de estudio que promueve LingQ se asemeja a los valores que defiende Krashen en sus cinco hipótesis sobre la adquisición de idiomas.

\subsection{Learning with Texts}

En 2010, Francisco Javier Martínez Lago, aún siendo estudiante, comenzó a desarrollar una alternativa a LingQ, llamada \textit{Learning with Texts} o, por sus siglas, LWT. \autocite{LWT}

Este \textit{software} se desarrolló en PHP. Mantiene las funcionalidades básicas de LingQ, pero tiene algunos cambios sustanciales en cuanto a la experiencia de usuario. En primer lugar, en lugar de accederse mediante internet, esta aplicación se instala en el ordenador del usuario. Además, mientras que LingQ solo permite el estudio de una selección limitada de idiomas, LWT ofrece la opción de configurar manualmente el comportamiento del idioma, de forma que se puede utilizar sea cual sea la lengua que desea adquirir el usuario.

Además, debido a que la aplicación se liberó a dominio público por decisión del autor, han surgido numerosos \textit{forks}, alguno de los cuales llega a ser más popular que la versión original, como

Aunque no se ha encontrado literatura científica que trate particularmente sobre LWT, al funcionar de manera tan similar a LingQ, las conclusiones obtenidas en los artículos previamente referenciados aplican también en este caso.

\subsection{Lute}

La aplicación Lute, creada y desarrollada principalmente por Jeff Zohrab, nació como un \textit{fork} de LWT. Sin embargo, desde su versión 3, todo el código se migró a Python. \autocite{Lute}

La interfaz de usuario de Lute es bastante similar a la de LWT incluso después de la refactorización. \todo[Comparar ambos]

\subsection{Relación con este trabajo}

Como se observa, existen varias soluciones con unas ideas y experiencias para el usuario muy similares aparentemente. Sin embargo, todas ellas tienen puntos en que podrían mejorar.

Se ha mencionado que Learning with Texts y Lute permiten estudiar cualquier idioma al usuario. Matizando algo más, permiten configurar los parámetros de cada idioma manualmente. Sin embargo, estos parámetros no son suficientes. Por ejemplo, debido a que estas herramientas procesan el texto separando palabras mediante espacios, los estudiantes de idiomas como el japonés, que no utiliza espacios, necesitarán utilizar aplicaciones de terceros para adaptar de antemano sus contenidos.

Además, existen funciones de LingQ que no han podido ser replicados en las alternativas de código abierto:
\begin{enumerate}
	\item El autocompletado de la definición de las palabras consultadas es una tarea que parece dificultarse todavía en Lute. Esto es debido a que el autocompletado en LingQ depende de las definiciones que otros usuarios han introducido manualmente. Como en Lute no existe una comunidad en línea, no existe quien introduzca las definiciones antes que el usuario.
	\item La instalación tanto de LWT como de Lute requiere conocimientos relativamente avanzados sobre informática, como la creación de una base de datos MySQL o ejecutar \textit{scripts} de Python dentro de un entorno, respectivamente. Aunque desde el punto de vista del desarrollador pueda dar la sensación de que son requisitos tan simples como seguir un tutorial, tener un proceso de instalación tedioso para personas sin experiencia digital supone que probablemente nunca lleguen a acceder al programa. \citationNeeded
\end{enumerate}

En cuanto a Lute, aunque la experiencia de usuario es mayormente más positiva que en el caso de LWT, sobre todo por sus tiempos de carga, todavía existen puntos de carencia, como el proceso de actualización, en el que el programa requiere que el usuario exporte todos los datos de la aplicación y los vuelva a importar después de la actualización, en lugar de actualizar automáticamente la base de datos sin suponer un esfuerzo por parte del estudiante.

También se pueden plantear dudas sobre su priorización de tareas. Por ejemplo, la aplicación tiene implementado un sofisticado sistema para almacenar imágenes en el campo de definiciones. Esta funcionalidad supone un coste de mantenimiento en el largo plazo, y la literatura científica no parece respaldar su utilidad, ya que el método de lectura extensiva es especialmente útil en estudiantes con un nivel intermedio o avanzado \citationNeeded, los cuales deberán aprender palabras demasiado concretas como para poder entender su significado mediante una imagen.

En definitiva, si bien cada solución tiene algunas ventajas frente a la anterior, todavía quedan numerosos frentes abiertos. Así, el objetivo de este Trabajo de Fin de Grado es explorar estos frentes y proponer una aplicación alternativa que los solucione sin sacrificar funcionalidades esenciales.
