\chapter{Estado de la cuestión}  
\addcontentsline{toc}{chapter}{\numberline{}Estado de la cuestión}

\section{Marco teórico del trabajo}

En el siglo XX varios lingüistas se interesaron por la pregunta sobre la forma en que los adultos adquieren una segunda lengua \autocite{Ellis_2021}. 

Una de las aportaciones más significantes en la historia del estudio sobre la adquisición de la segunda lengua es el modelo monitor o la hipótesis de entrada\footnote{El Dr. Krashen originalmente bautizó como \textit{hipótesis de entrada} a una sola de las cinco hipótesis. Aunque el uso popular favorezca el uso del término \textit{hipótesis de entrada} para todo el conjunto, en el presente documento se lo denominará \textit{modelo monitor} para evitar confusiones.} propuesta por el lingüista Stephen Krashen \autocite{Krashen2003}. El modelo monitor es en realidad un conjunto de cinco hipótesis que ponen especial énfasis en la importancia de absorber contenido en la lengua objetivo, a la vez que consideran que la producción de contenido solo produce beneficios marginales \autocite{Krashen2003}. Dichas hipótesis son las siguientes:
\begin{itemize}
	\item \textbf{La hipótesis de adquisición-aprendizaje}: Dice que existen dos formas independientes de desarrollar las habilidades en un idioma. Por una parte, está el \textit{aprendizaje}, un proceso consciente: ocurre prestando atención al idioma, \textit{estudiando}, por ejemplo, sus reglas gramaticales Por otra parte, está la \textit{adquisición}, un proceso subconsciente, lo cual significa que la persona no se da cuenta de que aumentan sus habilidades.
	\item \textbf{La hipótesis del orden natural}: Dice que las distintas partes de un idioma se adquieren en un orden predecible. Krashen también describe tres propiedades de esta previsibilidad:
	\begin{enumerate}
		\item El orden no está dado por la complejidad de los conceptos a adquirir; puede ser que un ítem gramatical complejo suela adquirirse antes que otro simple. Esto dificulta el diseño de los libros de texto que no respetan el orden natural.
		\item El orden natural es inmutable, lo que significa que hacer más ejercicios o leer explicaciones más claras sobre un concepto gramatical no supone que vaya a adquirirse más rápido.
		\item El orden natural no es el que debería respetarse a la hora de enseñar un idioma, porque... \todo
	\end{enumerate}
	\item \textbf{La hipótesis del monitor}: Krashen se refiere como \textit{monitor} a la conciencia del hablante. La idea es la siguiente: antes de producir lenguaje, el cerebro filtra, de manera consciente, la frase, buscando errores gramaticales que pueden ser corregidos antes o después de pronunciar las palabras. Según esta hipótesis, esta \textit{monitorización} es la única función que tiene el pensamiento conciente, y en el largo plazo los beneficios que trae a la fluidez son marginales, resultando realmente útil tan solo en situaciones en que la mera adquisición de la lengua no sea suficiente para que el usuario interiorice por completo algunas propiedades del idioma.
	
	Krashen además explica que se deben cumplir tres condiciones para que pueda utilizarse el monitor: el hablante debe conocer la herramienta, conocer las reglas gramaticales del idioma, y contar con tiempo de aplicarla. Es poco probable que estas tres condiciones se den: sobre todo la tercera, pues la demora que conlleva aplicar este método ralentiza a niveles indeseados la conversación. Este es el motivo por el que Krashen recomienda utilizarlo en conversaciones de por sí lentas, o en la escritura.
	
	\item \textbf{La hipótesis de la entrada}: Esta hipótesis trata de responder a la pregunta de cómo la adquisición de un idioma ocurre realmente. La respuesta que propone es que es la comprensión de un texto la que produce la adquisición.
	
	Esta hipótesis depende en cierta medida de la hipótesis del orden natural, la cual, recuérdese, indica que existen elementos de un idioma que los estudiantes adquieren antes que otros. Se puede formular, entonces, que según los elementos del idioma que el estudiante ha adquirido hasta el momento, se le puede asignar un nivel numérico, como $i$. El valor de $i$ no es importante por sí mismo; tan solo es una nomenclatura para indicar que este nivel es más avanzado que otro hipotético nivel $i - 1$, y que también existe un nivel más avanzado como $i + 1$.
	
	Con esta reformulación, la hipótesis de entrada explica que la forma en que el estudiante pasa de tener un nivel $i$ a tener un nivel $i + 1$ es comprendiendo contenidos con elementos del nivel $i + 1$.
	
	Esto es posible de diversas maneras, como un estudio activo y consciente de dichos elementos, o presentar el idioma junto con contenidos extralingüísticos, como imágenes.
	
	Krashen, además, asegura que la hipótesis de la entrada funciona por causa-efecto. Es decir: si una persona absorbe y comprende contenido de nivel $i + 1$, la misma inevitablemente acabará adquiriendo el nivel $i + 1$.
	
	También predica que la mera producción de la lengua (es decir, hablar o escribir en el otro idioma) no afecta directamente a la adquisición de la lengua. Aun así, indirectamente sí puede ser de ayuda, ya que los efectos psicológicos y sociales de ser un hablante de la lengua facilitan tomar provecho de oportunidades para absorberla.
	
	\item \textbf{La hipótesis del filtro afectivo}: Esta hipótesis es una excepción al primer colorario de la hipótesis de la entrada. Según \todo, el cerebro contiene una parte denominada \textit{dispositivo de la adquisición de idiomas}. Este dispositivo, que se encarga de la adquisición, puede ser bloqueado por el \textit{filtro afectivo}, causado por una situación psicológica desfavorable, como baja autoestima o ansiedad.
\end{itemize}

El modelo monitor fue propuesto hace al rededor de cinco décadas. Por supuesto, a lo largo de estos últimos años, se han realizado innumerables aportes al estudio sobre la adquisición de la segunda lengua que, si bien no contradicen del todo las ideas arriba mencionadas, sí expanden sobre ellas y aclaran matices importantes \autocite{https://doi.org/10.1111/flan.12552}. \todo[¿Cuáles?]

En definitiva, aunque se han realizado modificaciones a la propuesta original de Krashen, los pilares de su teoría no han sido refutados de momento y, al contrario, han sido respaldados con pruebas empíricas.

\todo[Uso de los dispositivos electrónicos para el estudio de idiomas]

\section{Trabajos relacionados}

\subsection{LingQ}

\subsection{LWT}

\subsection{Lute}

\subsection{Anki}

\subsection{Duolingo}
