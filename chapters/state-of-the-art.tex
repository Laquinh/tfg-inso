\chapter{Estado de la cuestión}  
\addcontentsline{toc}{chapter}{\numberline{}Estado de la cuestión}

\section{Marco teórico del trabajo}

En el siglo XX varios lingüistas se interesaron por la pregunta sobre la forma en que los adultos adquieren una segunda lengua \autocite{Ellis_2021}. 

Una de las aportaciones más significantes en la historia del estudio sobre la adquisición de la segunda lengua es el modelo monitor o la hipótesis de entrada\footnote{El Dr. Krashen originalmente bautizó como \textit{hipótesis de entrada} a una sola de las cinco hipótesis. Aunque el uso popular favorezca el uso del término \textit{hipótesis de entrada} para todo el conjunto, en el presente documento se lo denominará \textit{modelo monitor} para evitar confusiones.} propuesta por el lingüista Stephen Krashen \autocite{Krashen2003}. El modelo monitor es en realidad un conjunto de cinco hipótesis que ponen especial énfasis en la importancia de absorber contenido en la lengua objetivo, a la vez que consideran que la producción de contenido solo produce beneficios marginales \autocite{Krashen2003}. Dichas hipótesis son las siguientes:
\begin{itemize}
	\item \textbf{La hipótesis de adquisición-aprendizaje}: Dice que existen dos formas independientes de desarrollar las habilidades en un idioma. Por una parte, está el \textit{aprendizaje}, un proceso consciente: ocurre prestando atención al idioma, \textit{estudiando}, por ejemplo, sus reglas gramaticales Por otra parte, está la \textit{adquisición}, un proceso subconsciente, lo cual significa que la persona no se da cuenta de que aumentan sus habilidades.
	\item \textbf{La hipótesis del orden natural}: Dice que las distintas partes de un idioma se adquieren en un orden predecible. Krashen también describe tres propiedades de esta previsibilidad:
	\begin{enumerate}
		\item El orden no está dado por la complejidad de los conceptos a adquirir; puede ser que un ítem gramatical complejo suela adquirirse antes que otro simple. Esto dificulta el diseño de los libros de texto que no respetan el orden natural.
		\item El orden natural es inmutable, lo que significa que hacer más ejercicios o leer explicaciones más claras sobre un concepto gramatical no supone que vaya a adquirirse más rápido.
		\item El orden natural no es el que debería respetarse a la hora de enseñar un idioma, porque... \todo
	\end{enumerate}
	\item \textbf{La hipótesis del monitor}: 
	\item \textbf{La hipótesis de la entrada}:
	\item \textbf{La hipótesis del filtro afectivo}:
\end{itemize}

El modelo monitor fue propuesto hace al rededor de cinco décadas. Por supuesto, a lo largo de estos últimos años, se han realizado innumerables aportes al estudio sobre la adquisición de la segunda lengua \autocite{https://doi.org/10.1111/flan.12552}.

En definitiva, aunque se han realizado modificaciones a la propuesta original de Krashen, los pilares de su teoría no han sido refutados de momento y, al contrario, han sido respaldados con pruebas empíricas.

Características de aplicaciones para estudio de idiomas. \todo

\section{Trabajos relacionados}

\subsection{LingQ}

\subsection{LWT}

\subsection{Lute}

\subsection{Anki}

\subsection{Duolingo}
